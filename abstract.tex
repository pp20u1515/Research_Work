\chapter*{\hfill{\centering РЕФЕРАТ}\hfill}
\addcontentsline{toc}{chapter}{РЕФЕРАТ}
\setcounter{page}{2}

Научно -- иследовательская работа 19 с., 3 рис., 1 табл., 7 ист.

БЛОКИРОВКИ В БАЗЕ ДАННЫХ, ТРАНЗАКЦИИ, УРОВНИ ИЗОЛЯЦИИ, МНОГОПОЛЬЗОВАТЕЛЬСКАЯ СРЕДА, СУБД, ПРОИЗВОДИТЕЛЬНОСТЬ.

Объект исследования -- методы блокировки объектов в реляционных базах данных.

Целю работы является предоставить практические решения для обеспечения бесперебойной работы баз данных в условиях многопользовательской среды, где конфликты и взаимные блокировки могут оказать серьезное воздействие на результативность и надежность системы.

В данной исследовательской работе были рассмотрены методы управления блокировками в реляционных базах данных, охватывающие уровни строки, столбцов и страниц, а также включающие в себя блокировки намерений, механизмы обнаружения и разрешения взаимных блокировок. 
Также проведен анализ воздействия различных уровней изоляции на производительность и согласованность данных.

Результат данной работы показал, что каждый метод имеет ряд преимуществ и недостаткам и применим в зависимости от требований и ограничений к задаче.