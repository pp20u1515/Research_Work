\chapter{Взаимоблокировка в многозадачных средах: детали, детекция и методы разрешения}

Наиболее распространенная форма взаимоблокировки возникает, когда два или более потоков ожидают ресурса, который принадлежит другому потоку\cite{deadlock_detection}. Это проиллюстрировано следующим образом:
\FloatBarrier
\begin{table}[h]
	\renewcommand{\arraystretch}{1.8}
	\renewcommand{\tabcolsep}{0.1cm} 
	\begin{center}
		\begin{tabular}{|c|c|} 
			\hline
			Поток1 & Поток2 \\
			\hline
			Принимает блокировку A & Принимает блокировку B \\
			\hline 
			Блокировка запросов B & Запросы блокируют A \\
			\hline
		\end{tabular}
	\end{center}
\end{table}
\FloatBarrier
Если обе последовательности выполняются одновременно, поток 1 никогда не получит блокировку B, так как она принадлежит потоку 2, а поток 2 никогда не получит блокировку A, так как принадлежит потоку 1. В лучшем случае это приводит к остановке участвующих потоков, а в худшем — к остановке системы.
Реляционная система управления базами данных может следить за тем, какие транзакций блокируют какие ресурсы и какие запросы на разблокировку они выполняют.

После обнаружения взаимных блокировок, система должна разрешить их чтобы продолжить выполнение транзакций. Описаны несколько методов разрешения блокировок, такие как:
\begin{itemize}
	\item автоматическое разрешение: СУБД может автоматически определить какую из блокировок следует откатить, чтобы разрешить конфликт и продолжить выполнение транзакцией;
	\item ручное разрешение: администратор базы данных может вмешаться и вручную откатить определенную транзакцию, чтобы разрешить блокировку.
\end{itemize}

Для ручного управления блокировками в реляционных базах данных существует механизм уровня изоляции, который присваивается нужной транзакции и выдает при её работе соответствующие блокировки.

Виды уровни изоляции:
\begin{itemize}
	\item ReadUncommitted;
	\item ReadCommitted;
	\item Snapshot;
	\item RepeatableRead;
	\item Serializable.
\end{itemize}

ReadUncommitted – полное отсутствие исключительных и разделяемых блокировок. Максимальная производительность работы с такими данными, но они могут быть изменены любой транзакцией и вызвать недействительное чтение.

ReadCommitted – разделяемые блокировки присутствуют, пока считываются данные. 

Snapshot – с данных, к которым обращается транзакция, берется копия. Каждая транзакция может делать это повторно и работать только со своей копией данных.

RepeatableRead – разделяемые блокировки накладываются на все данные, с которыми работает транзакция.

Serializable – блокировки диапазона данных с которыми работает транзакция. Во время такой блокировки запрещены вставки и обновления записей другими пользователями.













