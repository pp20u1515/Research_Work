\chapter*{\hfill{\centering ЗАКЛЮЧЕНИЕ}\hfill}
\addcontentsline{toc}{chapter}{ЗАКЛЮЧЕНИЕ}

В результате исследования методы обнаружения и разрешения взаимных блокировок, а также анализа уровней изоляции в системах управления базами данных, становится ясным, что эти аспекты играют критическую роль в обеспечении эффективности и надежности работы с данными в многопользовательских сценариях.

Одним из ключевых выводов является необходимость постоянного совершенствования методов обнаружения взаимных блокировок, чтобы минимизировать негативное воздействие на производительность и обеспечить бесперебойную работу системы. 
Современные технологии исследований в области баз данных продолжают активно разрабатывать новые методы, направленные на повышение эффективности обнаружения подобных конфликтов.

В контексте уровней изоляции важно отметить, что выбор подходящего уровня изоляции зависит от конкретных требований консистентности данных и производительности приложения. 

От уровня READ UNCOMMITTED, обеспечивающего минимальную изоляцию, до SERIALIZABLE, обеспечивающего максимальную, каждый уровень предоставляет свой баланс между видимостью данных и защитой от взаимных блокировок.

Заключение исследования подчеркивает важность тщательного анализа требований приложения при выборе методов обнаружения блокировок и уровней изоляции. 
Осознанный и грамотный подход к этим аспектам способствует созданию устойчивых, высокопроизводительных и надежных систем управления базами данных в условиях динамичного многопользовательского окружения.

Кроме того, важно отметить, что блокировки баз данных играют важную роль в обеспечении целостности данных при параллельном доступе нескольких пользователей или приложений к общим данным.
Они помогают предотвратить конфликты доступа к данным и поддерживают целостность информации.

Блокировка данных ограничивает возможность другим пользователям или приложениям получать доступ или изменять данные во время выполнения операций обновления или изменения.
Однако использование блокировок на уровне таблицы может снизить производительность приложения из-за ограничения доступа ко всей таблице.

Понимание типов блокировок и их контекстов использования важно для эффективного управления к доступам данных и поддержания целостности базы данных в многопользовательской среде.

В ходе данной работе были:

\begin{enumerate}[label={\arabic*)}]
	\item проанализированы существующие методы управления блокировками, механизмы обеспечения целостности данных и методы уровней изоляции в реляционных базах данных;
	\item установлены требования и ограничения для обеспечения бесперебойной работы баз данных в многопользовательской среде, учитывая особенности конкретной системы;
	\item разработана стратегия управления блокировками, включая определение оптимальных уровней изоляции и применение соответствующих методов обнаружения и разрешения взаимных блокировок;
	\item подготовлен отчет по данной работе.
\end{enumerate}
